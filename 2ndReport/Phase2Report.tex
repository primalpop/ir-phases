\documentclass[paper=a4, fontsize=11pt]{scrartcl}
\usepackage[T1]{fontenc}
\usepackage{fourier}

\usepackage[english]{babel}															% English language/hyphenation
\usepackage[protrusion=true,expansion=true]{microtype}	
\usepackage{amsmath,amsfonts,amsthm} % Math packages
\usepackage[pdftex]{graphicx}	
\usepackage{url}


%%% Custom sectioning
\usepackage{sectsty}
\allsectionsfont{\centering \normalfont\scshape}


%%% Custom headers/footers (fancyhdr package)
\usepackage{fancyhdr}
\pagestyle{fancyplain}
\fancyhead{}											% No page header
\fancyfoot[L]{}											% Empty 
\fancyfoot[C]{}											% Empty
\fancyfoot[R]{\thepage}									% Pagenumbering
\renewcommand{\headrulewidth}{0pt}			% Remove header underlines
\renewcommand{\footrulewidth}{0pt}				% Remove footer underlines
\setlength{\headheight}{13.6pt}


%%% Equation and float numbering
\numberwithin{equation}{section}		% Equationnumbering: section.eq#
\numberwithin{figure}{section}			% Figurenumbering: section.fig#
\numberwithin{table}{section}				% Tablenumbering: section.tab#


%%% Maketitle metadata
\newcommand{\horrule}[1]{\rule{\linewidth}{#1}} 	% Horizontal rule

\title{
		%\vspace{-1in} 	
		\usefont{OT1}{bch}{b}{n}
		\normalfont \normalsize \textsc{Information Retrieval} \\ [25pt]
		\horrule{0.5pt} \\[0.4cm]
		\huge Programming Project - Phase 2 \\
		\horrule{2pt} \\[0.5cm]
}
\author{
		\normalfont 								\normalsize
        Primal Pappachan\\[-3pt]		\normalsize
        primal1@umbc.edu\\[-3pt]		\normalsize
        \today
}
\date{}


%%% Begin document
\begin{document}
\maketitle
\section{Introduction}
Lorem ipsum dolor sit amet, consectetuer adipiscing elit. Aenean commodo ligula eget dolor. Aenean massa. Cum sociis natoque penatibus et magnis dis parturient montes, nascetur ridiculus mus. Donec quam felis, ultricies nec, pellentesque eu, pretium quis, sem. In enim justo, rhoncus ut, imperdiet a, venenatis vitae, justo. Nullam dictum felis eu pede mollis pretium. Integer tincidunt. Cras dapibus. Vivamus elementum semper nisi. Aliquam lorem ante, dapibus in, viverra quis, feugiat a, tellus:
\begin{align} 
	\begin{split}
	(x+y)^3 	&= (x+y)^2(x+y)\\
					&=(x^2+2xy+y^2)(x+y)\\
					&=(x^3+2x^2y+xy^2) + (x^2y+2xy^2+y^3)\\
					&=x^3+3x^2y+3xy^2+y^3
	\end{split}					
\end{align}
Phasellus viverra nulla ut metus varius laoreet. Quisque rutrum. Aenean imperdiet. Etiam ultricies nisi vel augue. Curabitur ullamcorper ultricies 

\paragraph{Input}

\paragraph{Output}

\section{Preprocessing}

Tokenization improvements

to object oriented

to RegExp tokenize http://www.nltk.org/api/nltk.tokenize.html
http://docs.huihoo.com/nltk/0.9.5/guides/tokenize.html

to HTMLParser

\section{Term weighting}

tf-idf - algorithm and formula, hashtable

\subsection{BM25 Ranking}

formula - values of K1 and S1

\section{Evaluation}

Show two examples of input documents and the resulting (surviving) tokens and their weights. Discuss the efficiency of your approach in terms of your timings on a linux machine (use the time command) as a function of number of documents indexed. 

\begin{thebibliography}{1}

  \bibitem{notes} John W. Dower {\em Readings compiled for History
  21.479.}  1991.

  \bibitem{impj}  The Japan Reader {\em Imperial Japan 1800-1945} 1973:
  Random House, N.Y.

  \bibitem{norman} E. H. Norman {\em Japan's emergence as a modern
  state} 1940: International Secretariat, Institute of Pacific
  Relations.

  \bibitem{fo} Bob Tadashi Wakabayashi {\em Anti-Foreignism and Western
  Learning in Early-Modern Japan} 1986: Harvard University Press.

  \end{thebibliography}

%%% End document
\end{document}