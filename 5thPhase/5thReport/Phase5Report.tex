\documentclass[paper=a4, fontsize=11pt]{scrartcl}
\usepackage[T1]{fontenc}

\usepackage[english]{babel}															% English language/hyphenation
\usepackage[protrusion=true,expansion=true]{microtype}	
\usepackage{amsmath,amsfonts,amsthm} % Math packages

\usepackage{graphicx,subfigure}
%\usepackage[pdftex]{graphicx}	
\usepackage{url}
\usepackage{soul,color}

%%% Custom sectioning
\usepackage{sectsty}
\allsectionsfont{\centering \normalfont\scshape}


%%% Custom headers/footers (fancyhdr package)
\usepackage{fancyhdr}
\pagestyle{fancyplain}
\fancyhead{}											% No page header
\fancyfoot[L]{}											% Empty 
\fancyfoot[C]{}											% Empty
\fancyfoot[R]{\thepage}									% Pagenumbering
\renewcommand{\headrulewidth}{0pt}			% Remove header underlines
\renewcommand{\footrulewidth}{0pt}				% Remove footer underlines
\setlength{\headheight}{13.6pt}


%%% Equation and float numbering
\numberwithin{equation}{section}		% Equationnumbering: section.eq#
\numberwithin{figure}{section}			% Figurenumbering: section.fig#
\numberwithin{table}{section}				% Tablenumbering: section.tab#


%%% Maketitle metadata
\newcommand{\horrule}[1]{\rule{\linewidth}{#1}} 	% Horizontal rule

\title{
		%\vspace{-1in} 	
		\usefont{OT1}{bch}{b}{n}
		\normalfont \normalsize \textsc{Information Retrieval} \\ [25pt]
		\horrule{0.5pt} \\[0.4cm]
		\huge Programming Project - Phase 5 \\
		\horrule{2pt} \\[0.5cm]
}
\author{
		\normalfont 								\normalsize
        Primal Pappachan\\[-3pt]		\normalsize
        primal1@umbc.edu\\[-3pt]		\normalsize
        \today
}
\date{}


%%% Begin document
\begin{document}
\maketitle
\section{Introduction}
In Programming Assignment 5, I have analysed the 504-document HTML corpus by using document clustering. I have used my code from earlier phases of the project for tokenizing, calculating normalized term weights and building a term document matrix. I have used Python to code the entire program. To execute the program from a terminal (after setting right permissions for the file), type 

\begin{verbatim}
$./cluster files 
\end{verbatim}

For example
\begin{verbatim}
$./cluster ../input_files 
\end{verbatim}

The commandline parameter is path to the input files directory. You need to install the NLTK to run the program. Please refer to the documentation\footnote{\url{http://www.nltk.org/install.html}} on how to install NLTK.

\paragraph{Output}

The output is written to the file \textit{cluster.txt}. For every merge between clusters or documents, a line was written into the file as, \textit{Merging cluster1 and cluster2 into new_cluster}. Only the first 100 lines of output has been added to cluster.txt as per problem requirement.

\section{Implementation}

I extended the program from previous phases for implementing document clustering. The following sections outlines the data structures and algorithms used in the program.

\subsection{Data Structures}

Term Document Matrix

Similarity Matrix

Active Clusters

Cluster Info

\subsection{Similarity Matrix}

Cosine Similarity on Term Document Matrix

\subsection{Group Link Average}

Formula used

\subsection{Document Clustering}

Mention about 0.4

\section{Evaluation}

\begin{enumerate}
\item Which pair of HTML documents is the most similar? 

Document 434.html and 435.html are most similar as they were first to be merged into a cluster.

\item Which pair of documents is the most dissimilar?
 

\item Which document is the closest to the corpus centroid?

\end{enumerate} 

\clearpage

\begin{thebibliography}{1}

\bibitem{umbc} UMBC link

\bibitem{clus} Stanford link

\bibitem{clus1} UC Davis link

\end{thebibliography}

%%% End document
\end{document}